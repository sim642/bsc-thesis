\documentclass{beamer}

% copy-paste from mystyle...
\usepackage[utf8]{inputenc}

\usepackage[english, estonian]{babel} %the thesis is in Estonian

\renewcommand<>{\emph}[1]{{\only#2{\em}#1}} % https://tex.stackexchange.com/a/274385
\renewcommand{\em}{\bfseries} % always bold emphasis


\providecommand{\mytitle}{}
\newcommand{\myauthor}{Simmo Saan}

\addto\captionsestonian{
	\renewcommand{\mytitle}{Abstraktsete domeenide omaduspõhine testimine}
}

\addto\captionsenglish{
	\renewcommand{\mytitle}{Property-based Testing of Abstract Domains}
}

\let\sqleq\sqsubseteq
\let\sqgeq\sqsupseteq
\let\sqlt\sqsubset
\let\sqgt\sqsupset

\newcommand{\powerset}[1]{\mathcal{P}(#1)}
\let\emptyset\varnothing

\newcommand{\Var}{\mathsf{Var}}
\newcommand{\Val}{\mathsf{Val}}

\newcommand{\tf}{\mathit{tf}}

\newcommand{\ingl}[1]{ingl. \textit{\foreignlanguage{english}{#1}}}

% prefix tilde for approximate numbers - https://tex.stackexchange.com/a/9372
\newcommand{\presim}{{\raise.17ex\hbox{$\scriptstyle\sim$}}}


\title[Domeenide testimine]{\mytitle}
\subtitle[]{Bakalaureusetöö}
\author{\myauthor}
\institute[]{Tartu Ülikool, arvutiteaduse instituut}
\date{Juuni, 2018} % TODO täpne kuupäev

\usetheme{Madrid}

\begin{document}

\frame{\titlepage}

\begin{frame}{Ülesehitus}
	\tableofcontents
\end{frame}

\section{Sissejuhatus}
\begin{frame}{Eesmärk}

\end{frame}

\begin{frame}{Intervallid}
\begin{itemize}[<+->]
	\item Täisarvude staatiliseks analüüsiks saab kasutada \emph{intervalle}
	\begin{itemize}[<.->]
		\item Näiteks $[0, 3],\; [-1, 5],\; [2, 2],\; [1, +\infty],\; [-\infty, +\infty]$
	\end{itemize}

	\item Aritmeetilised tehted intervallidel
	\begin{itemize}[<.->]
		\item Näiteks liitmine $[0, 3] + [-1, 5] = [-1, 8]$
	\end{itemize}

	\item Osalise järjestuse seos sisalduvuse kaudu
	\begin{itemize}[<.->]
		\item Näiteks $[2, 2] \sqleq [0, 3] \sqleq [-1, 5] \sqleq [-\infty, +\infty]$
		\item Kokkuleppeliselt väiksem tähendab täpsemat
	\end{itemize}

	\item Ühendamise tehe ühendi kaudu
	\begin{itemize}[<.->]
		\item Näiteks $[0, 3] \sqcup [5, 7] = [0, 7]$
	\end{itemize}

	\item Suurim intervall
	\begin{itemize}[<.->]
		\item $\top = [-\infty, +\infty]$
	\end{itemize}
\end{itemize}
\end{frame}

\section{Teoreetiline taust}
\frame{}

\section{Goblint analüsaator}
\frame{}

\section{Testimise tulemused}
\frame{}

\section{Kokkuvõte}
\frame{}

\end{document}