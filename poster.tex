\documentclass{beamer}


% copy-paste from mystyle...
\usepackage[utf8]{inputenc}

\usepackage{inconsolata} %nicer font for code listings. (Use \ttfamily for lstinline bastype)

\usepackage[english, estonian]{babel} %the thesis is in Estonian

% widen, narrow - https://tex.stackexchange.com/a/203931, https://tex.stackexchange.com/q/50057
% this looks weird with partially thicker overlapping lines...
\usepackage{scalerel}
\def\widen{\mathrel{\ThisStyle{\stretchrel*{\ooalign{%
  \raise0.2\LMex\hbox{$\SavedStyle\sqcup$}\cr%
  \raise-0.2\LMex\hbox{$\SavedStyle\sqcup$}}}{\sqcup}}}}
\def\narrow{\mathrel{\ThisStyle{\stretchrel*{\ooalign{%
  \raise0.2\LMex\hbox{$\SavedStyle\sqcap$}\cr%
  \raise-0.2\LMex\hbox{$\SavedStyle\sqcap$}}}{\sqcap}}}}

\usepackage{tabu}   % Wide lines in tables
\usepackage{multirow}

\usepackage[outputdir=build]{minted}
\setminted{
	autogobble=true,
	tabsize=4
}
% \RecustomVerbatimEnvironment{Verbatim}{BVerbatim}{} % center all minted - https://tex.stackexchange.com/a/161128
\usepackage{xpatch,letltxmacro}
\LetLtxMacro{\bminted}{\minted}
\let\endbminted\endminted
\xpretocmd{\bminted}{\RecustomVerbatimEnvironment{Verbatim}{BVerbatim}{}}{}{}
% BVerbatim removes margins (https://tex.stackexchange.com/q/142444) and doesn't support frame

% If you really want to draw figures in LaTeX use packages tikz or pstricks
% However, getting a corresponding illustrations is really painful
\usepackage{tikz}
\usetikzlibrary{positioning,automata,arrows,fit,calc}

\renewcommand<>{\emph}[1]{{\only#2{\em}#1}} % https://tex.stackexchange.com/a/274385
\renewcommand{\em}{\bfseries} % always bold emphasis

\usepackage{comment}

\usepackage{appendixnumberbeamer}

\usepackage{pifont}
\DeclareUnicodeCharacter{2713}{\ding{51}}
\DeclareUnicodeCharacter{2717}{\hspace{1pt}\ding{55}\hspace{1pt}} % TODO: fucking hack

\usepackage{xcolor}


\providecommand{\mytitle}{}
\newcommand{\myauthor}{Simmo Saan}

\addto\captionsestonian{
	\renewcommand{\mytitle}{Abstraktsete domeenide omaduspõhine testimine}
}

\addto\captionsenglish{
	\renewcommand{\mytitle}{Property-based Testing of Abstract Domains}
}

\let\sqleq\sqsubseteq
\let\sqgeq\sqsupseteq
\let\sqlt\sqsubset
\let\sqgt\sqsupset

\newcommand{\powerset}[1]{\mathcal{P}(#1)}
\let\emptyset\varnothing

\newcommand{\Var}{\mathsf{Var}}
\newcommand{\Val}{\mathsf{Val}}

\newcommand{\tf}{\mathit{tf}}

\newcommand{\ingl}[1]{ingl. \textit{\foreignlanguage{english}{#1}}}

% prefix tilde for approximate numbers - https://tex.stackexchange.com/a/9372
\newcommand{\presim}{{\raise.17ex\hbox{$\scriptstyle\sim$}}}
\newcommand{\emptycolumn}[1]{\begin{column}{#1}\end{column}} % beamer columns must be manually padded to center...


\title[Domeenide testimine]{\mytitle}
\subtitle[]{Bakalaureusetöö}
\author{\myauthor}
\institute[]{Tartu Ülikool, arvutiteaduse instituut}
\date{Juuni, 2018} % TODO täpne kuupäev

\usetheme{Madrid}

\usepackage[orientation=portrait,size=a3,scale=1.75,debug]{beamerposter}

\begin{document}

\begin{frame}[fragile]
\maketitle

\vspace{-5ex}

\begin{columns}[t]
	\begin{column}{0.5\textwidth}
		\begin{block}{Staatiline programmianalüüs}
			\begin{itemize}
				\item Staatilise analüüsiga uuritakse programme ilma neid käivitamata. See võimaldab otsida vigu või tõestada teatud vigade puudumist.
				
				\item Üks võimalus on abstraktse interpretatsiooniga ligikaudsete programmi seisundite määramine. Vastav teooria garanteerib analüüsi korrektsuse (\ingl{sound}).
				
				\item Analüsaatorites endis esineb vigu, mistõttu tehtavad analüüsid ja nende korrektsus on rikutud.
			\end{itemize}
		\end{block}
		
		\begin{alertblock}{Eesmärk}
			Goblint analüsaatori
			\begin{itemize}
				\item Domeenide omaduspõhine testimine
				\item Vigade tuvastamine
			\end{itemize}
		\end{alertblock}
	\end{column}
	
	\begin{column}{0.4\textwidth}
		\begin{block}{Intervallid}
			\begin{itemize}
				\item Täisarvude staatiliseks analüüsiks saab kasutada \emph{intervalle}
				\begin{itemize}
					\item Näiteks $[0, 3],\; [-1, 5],\; [2, 2],\; [1, +\infty],\; [-\infty, +\infty]$
				\end{itemize}
			
				\item Aritmeetilised tehted intervallidel
				\begin{itemize}
					\item Näiteks liitmine $[0, 3] + [-1, 5] = [-1, 8]$
				\end{itemize}
			
				\item Osalise järjestuse seos sisalduvuse kaudu
				\begin{itemize}
					\item Näiteks $[2, 2] \sqleq [0, 3] \sqleq [-1, 5] \sqleq [-\infty, +\infty]$
					\item Kokkuleppeliselt väiksem tähendab täpsemat
				\end{itemize}
			
				\item Ühendamise tehe ühendi kaudu
				\begin{itemize}
					\item Näiteks $[0, 3] \sqcup [5, 7] = [0, 7]$
				\end{itemize}
			
				\item Suurim intervall
				\begin{itemize}
					\item $\top = [-\infty, +\infty]$
				\end{itemize}
			\end{itemize}
		\end{block}
	\end{column}
\end{columns}

\begin{columns}[t]
	\begin{column}{0.5\textwidth}
		\begin{block}{Näidisanalüüs intervallidega}
			\begin{columns}
				\emptycolumn{0.05\textwidth}
				\begin{column}{0.4\textwidth}
					\centering
					\begin{bminted}{c}
						int x = 0;
						while (x < 100)
							x++;
					\end{bminted}
				\end{column}
				\begin{column}{0.5\textwidth}
					\alert{Muutuja \texttt{x} väärtus}
			
					\centering
					\begin{tikzpicture}[
						->,>=stealth,
						node distance=0.75cm,
						every state/.style={inner sep=3pt, minimum size=0pt},
						stmt/.style={font=\ttfamily},
						initial text={},
						initial where=above
					]
						\node[initial,state,label=right:{{{\alert{$\bot$}}}}] (1) {1};
						\node[state,label={[yshift=0.3cm]right:{{{\alert{$[0, 100]$}}}}}] (2) [below=of 1] {2};
						\node[state,label=left:{{{\alert{$[0, 99]$}}}}] (3) [below left=of 2] {3};
						\node[accepting,state,label=right:{{{\alert{$[100, 100]$}}}}] (4) [below right=of 2] {4};
			
						\path
							(1) edge node[stmt,right] {x = 0} (2)
							(2) edge[bend right] node[stmt,above left,near start] {x < 100} (3)
							    edge node[stmt,above right] {x >= 100} (4)
							(3) edge[bend right] node[stmt,below right,very near start] {x++} (2);
					\end{tikzpicture}
				\end{column}
				\emptycolumn{0.05\textwidth}
			\end{columns}
		\end{block}
	\end{column}
	
	\begin{column}{0.4\textwidth}
		\begin{block}{Täielikud võred}
			Domeen (võimalike väärtuste hulk) peab moodustama \emph{täieliku võre}:
			\begin{itemize}
				\item Elementide hulk $\mathbb{D}$
				\item Osalise järjestuse seos $\sqleq$
				\item Ülemise raja tehe $\sqcup$
				\item Alumise raja tehe $\sqcap$
				\item Suurim element $\top$
				\item Vähim element $\bot$
			\end{itemize}
		\end{block}
	\end{column}
\end{columns}

\begin{columns}[t]
	\begin{column}{0.25\textwidth}
		\begin{block}{Omaduspõhine testimine}
			\begin{itemize}
				\item Haskellis QuickCheck
				\item Hea matemaatiliste tingimuste kontrollimiseks
				\item Omadused --- predikaadid
				\item Generaatorid --- juhuslikud elemendid
			\end{itemize}
		\end{block}
	\end{column}
	
	
	\begin{column}{0.2\textwidth}
		\begin{block}{Goblint analüsaator}
			\begin{itemize}
				\item TÜ, TUM
				\item Mitmelõimelised C programmid
				\item Andmejooksud
				\item Kirjutatud OCaml-is
			\end{itemize}
		\end{block}
	\end{column}
	
	\begin{column}{0.45\textwidth}
		\begin{block}{Goblinti täiendused}
		\begin{itemize}
			\item Domeenidesse generaatorid
		
			\item Kõik omadused omaduspõhiste testidena
			\item Näiteks ülemraja kommutatiivsus ($a \sqcup b = b \sqcup a$):
			
			\begin{minted}{ocaml}
make ~name:"join comm" (pair arb arb)
  (fun (a, b) ->
    D.equal (D.join a b) (D.join b a))
			\end{minted}
	
			kus \texttt{D} testitav domeen, \texttt{arb} selle generaator
		\end{itemize}
		\end{block}
	\end{column}
\end{columns}

\begin{columns}[t]
	\begin{column}{0.51\textwidth}
		\begin{block}{Ülevaatlikud tulemused}
			\begin{center}
				\begin{tabu}{cc|ccc}
				& & \multicolumn{3}{c}{Testide arv} \\
				Lähenemine & Võrdlemine & Kokku & Erindi teke & Mittekehtivad \\
				\hline
				\multirow{2}{*}{Alt üles} & \texttt{equal} & 468 & \presim 35 & \presim 75 \\
				%\cline{2-5}
				& \texttt{leq} & 468 & \presim 35 & \presim 69 \\
				\hline
				\multirow{2}{*}{Ülalt alla} & \texttt{equal} & 27 & 0 & \presim 12 \\
				%\cline{2-5}
				& \texttt{leq} & 27 & 0 & \presim 12
				\end{tabu}
			\end{center}
		\end{block}
	\end{column}
	
	\begin{column}{0.4\textwidth}
		\begin{block}{Erindi tekkimise ja mittekehtimisel põhjuseid}
			\begin{enumerate}
				\item Viga domeeni implementeerimisel
				\item Mittekehtimine teoreetilisel tasandil
				\item Teadlik ja dokumenteeritud mittekehtimine
				\item Sõltumine teistest probleemsetest omadustest/domeenidest
			\end{enumerate}
		\end{block}
	\end{column}
\end{columns}

\begin{columns}[t]
	\begin{column}{0.51\textwidth}
		\begin{block}{Trieri domeen}
			Trieri domeen on Goblintis vaikimisi kasutusel, kuid testimisel leiti selle disainist funamentaalne viga: ülemraja tehe pole assotsiatiivne.
			
			Goblinti arendajad võtavad leitut väga tõsiselt.
		\end{block}
	\end{column}
	
	\begin{column}{0.4\textwidth}
		\begin{alertblock}{Järeldus}
			Omaduspõhist testimist on võimalik efektiivselt rakendada abstraktsetest domeenidest vigade leidmiseks.
		\end{alertblock}
	\end{column}
\end{columns}

\end{frame}

\end{document}