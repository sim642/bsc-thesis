\documentclass[../thesis.tex]{subfiles}

\begin{document}

\section{Omaduspõhine testimine}
\TODO{Omaduspõhise testimise metoodika}
\TODO{Teek \enquote{qcheck}}

\subsection{Goblint}

\begin{figure}
	\centering
	\begin{minted}[mathescape]{ocaml}
		module type S =
		sig
		  type t (* domeeni elementide tüüp *)
		  val equal: t -> t -> bool (* seos $=$ *)
		  val leq: t -> t -> bool (* seos $\sqleq$ *)
		  val join: t -> t -> t (* tehe $\sqcup$ *)
		  val meet: t -> t -> t (* tehe $\sqcap$ *)
		  val bot: unit -> t (* element $\bot$ *)
		  val is_bot: t -> bool
		  val top: unit -> t (* element $\top$ *)
		  val is_top: t -> bool
		  val widen: t -> t -> t (* tehe $\widen$ *)
		  val narrow: t -> t -> t (* tehe $\narrow$ *)
		end
	\end{minted}
	\caption{Domeeni (lihtsustatud) signatuur Goblinti moodulis \texttt{Lattice}.}
	\label{fig:lattice-s}
\end{figure}

\begin{minted}{ocaml}
	val arbitrary: unit -> t QCheck.arbitrary
\end{minted}

\TODO{\textit{bottom-up}, \textit{top-down} lähenemised}
\TODO{Testimise tulemused}


\end{document}