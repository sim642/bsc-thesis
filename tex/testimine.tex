\documentclass[../thesis.tex]{subfiles}

\begin{document}

\section{Omaduspõhine testimine}
Omaduspõhine testimine (\ingl{property-based testing}) on programmide testimise metoodika, mis põhineb soovitud omaduste verifitseerimisel juhuslikult genereeritud testandmetel. Selle lähenemise populariseeris Haskelli teek nimega QuickCheck~\cite{claessen_quickcheck}, mis on nüüdseks implementeeritud ka paljude teiste programmeerimiskeelte jaoks ja mille kasutamine on funktsionaalprogrammeerimises laialdane.

Omaduspõhine testimine sobib hästi matemaatilist laadi omaduste, nagu eelmises peatükis toodud domeenide omaduste, kontrollimiseks, sest seda saab teha lihtsa tingimusega, vajamata sobivate testandmete väljamõtlemist. See võimaldab samu omadusi kontrollida kõigil domeenidel, sõltumata isegi nende elementide tüübist.

\subsection{Põhimõtted}
Testitavad omadused kirjeldatakse predikaatidena, millele omaduspõhise testimise raamistik genereerib hulga juhuslikke argumente, millel arvutab välja predikaadi väärtuse. Kui see on kõigil genereeritud argumentidel tõene, siis loetakse test läbituks ja omadus kehtivaks. Kui mõnel argumendil on predikaat väär, loetakse test läbikukkunuks ja omadus mittekehtivaks. Viimasel juhul kuvatakse probleemne argument ka testijale.

Omaduspõhise testimise raamistikes on tüüpiliselt juba defineeritud, kuidas vastava keele põhiliste andmetüüpide, sh paaride, järjendite jne, väärtusi juhuslikult genereerida. Puuduvate ja enda defineeritud andmetüüpide jaoks tuleb see ise juurde lisada. Seejuures \enquote{juhuslik} ei pea tähendama täielikult juhuslikku protsessi, vaid võib alati genereerida ka hulka huvitavaid ja sageli probleeme tekitavaid erijuhte (nt arv null, tühi järjend jne).

Juhuslikult genereeritud argumendid võivad olla \enquote{suured}: suured arvud, pikad sõned või järjendid, sügavad puud jne. Võib kergesti juhtuda, et mõne omaduse kontranäide on seetõttu inimese jaoks raskesti hoomatav ja jääb ebaselgeks, mis on selle argumendi juures erilist, et omadus ei kehti. Vigade analüüsimise lihtsustamiseks kasutatakse omaduspõhises testimises vääravate argumentide lihtsustamist ehk vähendamist (\ingl{shrinking}). Kirjeldatakse, kuidas leitud väärtusest saada lihtsamaid, mis oleks algsega piisavalt sarnased, et nende korral võiks omadus samuti mitte kehtida. Näiteks järjendist mingite elementide eemaldamine, et pikkus sellest väheneks. Vähendatud argumentidel väärtustatakse predikaat uuesti, leidmaks neid, mille korral omadus ikka ei kehti, ja rakendatakse vähendamist taas nii palju kordi kui võimalik, et lihtsustada probleemset argumenti maksimaalselt, muutes selle paremini hoomatavaks.

Tüüpiliselt on ka põhiliste andmetüüpide vähendaminie testimisraamistikesse juba sisse ehitatud, aga sedagi on võimalik ise laiendada oma andmetüüpidele.

\subsection{Goblint}
Goblint on mitmelõimeliste C programmide staatiline analüsaator, mille eesmärk on tõestada, et programmis puuduvad mitmelõimelisusega seotud vead (nt andmejooksud)~\cite{goblint2016}. Seda tehakse andmevooanalüüsi raamistikus, kasutades keerukamaid spetsiifilisi abstraktseid domeene. Goblinti autoritele on teada, et analüsaator ei käitu alati oodatud viisil, vaid võib teha vigu. Üheks võimalikuks probleemide allikaks on kasutusel olevad abstraktsed domeenid ja täpsemalt nende implementatsioonid. Leidmaks nende vigade allikat, testitigi käesoleva töö raames Goblintis implementeeritud domeenide omadusi.

Goblint ise on kirjutatud OCaml-is ja on avatud lähtekoodiga~\cite{goblint_repo}. Selle omaduspõhiseks testimiseks valiti vastav OCaml-i teek QCheck~\cite{qcheck_repo}, mis on sarnaste teekide hulgast enim kasutatud ja uuendatud. Kõigi töö raames tehtud muudatuste ja täienduse lähtekoodi leiab lisast~\ref{app:lahtekood}.

\begin{figure}
	\centering
	\begin{bminted}[mathescape]{ocaml}
		module type S =
		sig
		  type t (* domeeni elementide tüüp *)
		  val equal: t -> t -> bool (* seos $=$ *)
		  val leq: t -> t -> bool (* seos $\sqleq$ *)
		  val join: t -> t -> t (* tehe $\sqcup$ *)
		  val meet: t -> t -> t (* tehe $\sqcap$ *)
		  val bot: unit -> t (* element $\bot$ *)
		  val is_bot: t -> bool
		  val top: unit -> t (* element $\top$ *)
		  val is_top: t -> bool
		  val widen: t -> t -> t (* tehe $\widen$ *)
		  val narrow: t -> t -> t (* tehe $\narrow$ *)
		end
	\end{bminted}
	\caption{Domeeni (lihtsustatud) signatuur Goblinti moodulis \texttt{Lattice}.}
	\label{fig:lattice-s}
\end{figure}

Goblintis on domeenid implementeeritud OCaml-i moodulitena, mis muu hulgas sisaldavad domeeni elemendi tüüpi ning neil defineeritud seoseid ja tehteid, nagu näidatud joonisel~\ref{fig:lattice-s}. Domeenide omaduste testimiseks on esmalt vaja defineerida selle elementide generaator. Selleks lisati domeeni signatuuri funktsioon \texttt{arbitrary} signatuuriga:
\begin{minted}{ocaml}
	val arbitrary: unit -> t QCheck.arbitrary
\end{minted}
mille ainsaks argumendiks on ühiktüübi väärtus \texttt{()} ja mis tagastab \texttt{t QCheck.arbitrary} tüüpi väärtuse. Viimane hõlmab endas tüüpi \texttt{t} domeeni elementide generaatorit (\texttt{t QCheck.Gen}) ja vähendajat (\texttt{t QCheck.Shrink}) nii, nagu QCheck teek seda ette näeb.

Kõik omadused peatükkidest~\ref{sec:lattice-props} ja~\ref{sec:widen-narrow} implementeeriti omaduspõhiste testidena moodulis \texttt{DomainProperties}. Näiteks ülemraja kommutatiivsuse test näeb välja järgmiselt:
\begin{minted}{ocaml}
let join_comm = make ~name:"join comm" (pair arb arb)
                  (fun (a, b) -> D.equal (D.join a b) (D.join b a))
\end{minted}
kus \texttt{D} on testitava domeeni moodul ja \texttt{arb} on mugavuse mõttes \texttt{D.arbitrary ()}.

Testid implementeeriti polümorfselt OCaml-i funktorite abil, mis võimaldab samu omadusi kontrollida mistahes domeenil. See oligi eesmärk, sest kõik need omadused peavad universaalselt kehtima. Domeeni testimiseks peab sellel siiski olema defineeritud \texttt{arbitrary}, mida saab teha üksnes manuaalselt ja mis seega on kõige ajamahukam osa Goblinti arvukate domeenide testimise juures. Sellele läheneti kahelt poolt ning lõpuks sobivate generaatoritega varustatud domeenid on toodud lisas~\ref{app:goblint-domeenid}.

\subsection{Alt-üles lähenemine}
Esiteks testiti erinevaid täisarvudega töötavaid domeene moodulist \texttt{IntDomain}, mis on üsna sõltumatud ja seetõttu saab neid eraldiseisvalt testida. Täpse loetelu leiab lisa~\ref{app:goblint-domeenid} allosast. Paljud teised domeenid kasutavad oma elementidena konkreetsest analüüsitavast programmist leitud muutujaid, funktsioone vms, mis teeb nende testimise raskeks, sest pole selge, kuidas korrektselt genereerida neid juhuslikke elemente. Täisarvude abstraktsioonide puhul seda probleemi ei esine.

\paragraph{Täisarvudomeenide korrektsus}
Täisarve abstraheerivad domeenid omavad Goblintis tavalisetele domeenidele lisaks mitmeid spetsiifilisi funktsioone, nagu näidatud joonisel~\ref{fig:intdomain-s}.
Olemasolevatele lisaks implementeeriti lihtne täisarvude alamhulkade domeen \texttt{IntegerSet}, mille suhtes testiti ka kõigi täisarvude domeenide korrektsust, kasutades peatükis~\ref{sec:abstraction-props} toodud tingimusi.

Hulga abstraheerimiseks kasutati neis domeenides olevat \texttt{of\_int} funktsiooni, et üksikuid elemente abstraheerida ja seejärel ülemraja leida, mis vastaks kogu vaadeldavale arvuhulgale. See võimaldas testida kõigi tehete, nii ühekohaliste (nt \texttt{neg}) kui ka kahekohaliste (nt \texttt{add}), korrektsust alamhulkade kaudu defineeritute suhtes.

Kuna abstraktsiooni õigsuse testimiseks on domeenile lisaks vaja teist domeeni, mille abstraheerimist kontrollida, ja vastavaid tehteid mõlemas domeenis, siis sellist testimist üldisel tasandil teostada ei saa. Täisarvude domeenidel tehtu näitas, et selliseid omadusi on siiski võimalik testida, kuigi vajab olulist lisatööd.

\begin{figure}
	\centering
	\begin{bminted}[mathescape]{ocaml}
		module type S =
		sig
		  include Lattice.S (* sisaldab domeeni signatuuri, vt joonis $\texttt{\ref{fig:lattice-s}}$ *)
		  val of_int: int64 -> t (* funktsioon $\alpha$ ühe arvu jaoks *)
		  val neg: t -> t (* vastandarvu tehe, unaarne $-$ *)
		  val add: t -> t -> t (* liitmistehe, binaarne $+$ *)
		  val sub: t -> t -> t (* lahutamistehe, binaarne $-$ *)
		  (* ..., veel tehteid *)
		end
	\end{bminted}
	\caption{Täisarve abstraheeriva domeeni (lihtsustatud) signatuur Goblinti moodulis \texttt{IntDomain}.}
	\label{fig:intdomain-s}
\end{figure}

\paragraph{Käivitamine}
Nende domeenide testimiseks loodi moodul \texttt{Maindomaintest}, mida saab Goblintist endast eraldi kompileerida (käsuga \texttt{make domaintest}) ja käivitada (käsuga \texttt{./goblint.domaintest -v}). Selles moodulis on järjend domeenidest, mida käivitamisel eraldiseisvalt testitakse.

\subsection{Ülalt-alla lähenemine}
Teiseks testiti tervet domeeni, mida Goblint päriselt etteantud programmi analüüsimisel standardseadistuses kasutama hakkaks. See domeen pannakse sõltuvalt seadistusest kokku mitme eri analüüsi (Base, Escape, Mutex) domeenidest, mis on omakorda moodustatud paljudest domeenidest. Sellised sõltuvusi arvestades saab analüüsiks kasutatavat domeeni ette kujutada kui puud või sõltuvusgraafi (\ingl{dependency graph}). Seda ongi tehtud lisas~\ref{app:goblint-domeenid}, mille ülaosas on kujutatud neid tervikliku domeeni komponente, mida testiti.

Tervikliku domeeni elementide genereerimiseks oleks kaudselt vaja genereerida elemente kõigist teistest domeenidest, millest sõltutakse. See eeldab korraga kõigi Goblinti domeenide jaoks generaatorite lisamist, mis on väga töömahukas. Kui mingite domeenide generaatorid defineerida triviaalselt, näiteks alati $\bot$ genereerimise teel, siis on võimalik terve domeeni elemente genereerida, ilma et oleks vaja absoluutselt kõiki generaatoreid lisada. See on võimalik, kuna triviaalse generaatoriga domeen ei kutsu välja sõltuvuseks oleva domeeni generaatorit ja seetõttu viimane ei pea generaatorit üldse implementeerima. Domeenide Variables, Normal ja PMap generaatorid selliselt hetkel implementeeritigi.

\paragraph{Käivitamine}
Selliseks testimiseks lisati Goblinti käsurealippude hulka uus nimega \texttt{dbg.test.domain}. Kui see on sisse lülitatud, siis enne etteantud programmi analüüsimist jooksutatakse analüüsiks kasutataval domeenil\footnote{Täpsemalt \enquote{domeenidel}: Goblint kasutab analüüsis kahte erinevat domeeni, n-ö lokaalset ja globaalset.} samad testid. Goblinti näidiskäivitus koos selle lipuga toimub järgneva käsuga:
\begin{minted}{bash}
	./goblint --enable dbg.test.domain \
	    ./tests/regression/04-mutex/01-simple_rc.c
\end{minted}

\subsection{Raskused testimisel}
\paragraph{Implikatiivsed omadused}
Mõned peatükis~\ref{sec:lattice-props} toodud omadustest on implikatsiooni kujul, see tähendab, et omadus peab kehtima ainult siis, kui mingid eeldused on täidetud. Nende testimise keerukuseks on see, et juhuslikke domeeni elemente genereerides ei pruugi need eeldust rahuldada. Sel juhul on implikatsioon küll tõene, aga ei väida midagi sisulist.

Kõige enam on see probleemiks osalise järjestuse antisümmeetrilisuse testimisel, milleks peaks juhuslikult genereerima domeeni elemendid $a$ ja $b$ nii, et $a \sqleq b$ ja $b \sqleq a$, enne kui saab kontrollida nõutud tingimust $a = b$. See on äärmiselt ebatõenäoline, kui testitavas domeenis on võrdlemisi palju elemente. Näiteks selle omaduse 100 (või isegi 10000) kordsel testimisel üle 32-bitiste täisarvude eeldust rahuldavaid argumente enamasti ei genereerita, mistõttu antisümmeetrilisus jääb praktiliselt kontrollimata.

Sellele vaatamata on implikatiivsete omaduste testid siiski Goblintis implementeeritud, sest need võivad probleeme avastada, kui järjestusseoses on mõni suurem viga.

\paragraph{Võrdlemine}
Enamike omaduste testimisel on vaja kontrollida mingite domeeni elementide võrdumist. Esimese loomuliku variandina kasutati selleks domeenide \texttt{equal} funktsiooni (vt joonis~\ref{fig:lattice-s}). Kuna pole kindel, et kõigis Goblinti domeenides \texttt{equal} on defineeritud võre struktuuriga kooskõlaliselt, siis viidi samad testid läbi ka teisiti: testimisel kasutati elementide \texttt{a} ja \texttt{b} võrdumise tingimusena \texttt{leq a b \&\& leq b a}. Eeliseks on see, et selline võrdlus ei eelda, et \texttt{equal} funktsioon oleks kooskõlaliselt defineeritud, aga puuduseks see, et eeldatakse, et järjestus on korrektne.

Võrduse ja osalise järjestuse vahelise ebakõla peaks tuvastama osalise järjestuse antisümmeetrilisuse test, kuid eelmises punktis kirjeldatud põhjustel ei pruugi see välja tulla. Testimise käigus paistis selline ebakõla välja vaid domeeni \texttt{CircInterval} vähima ja suurima elemendi puhul (testides \enquote{bot is\_bot}, \enquote{top is\_top}) ning järjestuse ja rajade vahelise ekvivalentsi juures (testides \enquote{connect join}, \enquote{connect meet}).

\paragraph{Laiendamine}
Esialgse testimise tulemused näitasid, nagu oleks kõigis testitud domeenides laiendamise operaator vigane, sest ei rahulda selle ainsat omadust peatükist~\ref{sec:widen-narrow}. Osutub, et Goblinti domeenides olevat laiendamise operaatorit \texttt{widen} ei kasutata täpselt nii, nagu ülalkirjeldatud teooria seda ette näeb, vaid (ajaloolistel põhjustel) alati kujul \texttt{D.widen a (D.join a b)}, kus teiseks argumendiks on juba vastav ülemraja.

Seega nõutud omaduse $a \sqcup b \sqleq a \widen b$ asemel peaks kehtima omadus
\begin{align*}
	a \sqcup (a \sqcup b) &\sqleq a \widen (a \sqcup b) \qquad\stackrel{\text{assotsiatiivsus}}{\Longrightarrow} \\
	(a \sqcup a) \sqcup b &\sqleq a \widen (a \sqcup b) \qquad\stackrel{\text{idempotentsus}}{\Longrightarrow} \\
	a \sqcup b &\sqleq a \widen (a \sqcup b).
\end{align*}
Goblintis laiendamise operaatori testimiseks lõpuks kasutatigi viimast kuju. Seejuures eeldatakse, et ülemraja leidmise tehe on assotsiatiivne ja idempotentne, kuid neid kontrollitakse niikuinii vahetult omaette testidega.

\subsection{Tulemused}
\begin{table}
	\caption{Domeenide erinevatel meetoditel testimise tulemused.}
	\centering
	\begin{tabu}{cc|ccc}
	\hline
	 & & \multicolumn{3}{c}{Testide arv} \\
	Lähenemine & Võrdlemine & Kokku & Erindi teke & Mittekehtivad \\
	\hline
	\multirow{2}{*}{Alt-üles} & \texttt{equal} & 468 & \presim 35 & \presim 75 \\
	%\cline{2-5}
	 & \texttt{leq} & 468 & \presim 35 & \presim 69 \\
	\hline
	\multirow{2}{*}{Ülalt-alla} & \texttt{equal} & 27 & 0 & \presim 12 \\
	%\cline{2-5}
	 & \texttt{leq} & 27 & 0 & \presim 12 \\
	\hline
	\end{tabu}
	\label{tab:tulemused}
	\TODO{Uuenda kui muutunud}
\end{table}

Testide jooksutamise tulemused on toodud tabelis~\ref{tab:tulemused}, kus on toodud mõlema lähenemise tulemused. Kuna võrdlusmetoodika puhul on tegemist kompromissiga, siis viidi samad testid läbi mõlemal moel.

Mõnede omaduste testimisel tekkisid erindid enne kui kehtivust kontrollida oli võimalik. Enamus juhtudel seetõttu, et domeen ei moodusta täielikku võre ning elementide $\bot$ ja $\top$ asemel antakse erind. Sel põhjusel lõpevad erindiga domeeni \texttt{Integers} testid, mis kontrollivad vähima ja suurima elemendiga seotud omadusi.

Omaduste mittekehtimise põhjusteks võis olla:
\begin{enumerate}[nosep]
	\item Omaduse mittekehtimine teoreetilisel tasandil. Sel põhjusel ei kehti domeeni \texttt{Trier} ülemraja assotsiatiivsus.
	\item Omaduse dokumenteeritud mittekehtimine. Sel põhjusel tõenäoliselt ei kehti domeeni \texttt{CircInterval} rajade omadused, sest neid on nimetatud \enquote{pseudo-ülemrajaks} ja \enquote{pseudo-alamrajaks}.
	\item Omaduse sõltumine teistest mittekehtivatest omadustest samal või domeeni osaks oleval teisel domeenil. Sel põhjusel ei kehti domeeni \texttt{IntDomTuple} paljud omadused, kuna need ei kehti domeenidel, millest see koosneb.
	\item Viga domeeni implementeerimisel.
\end{enumerate}
Kuna töö eesmärgiks polnud vigade parandamine, siis põhjustesse oluliselt ei süvenetud, mistõttu ei saa kindlalt kõigi ebaõnnestunud testide kohta midagi väita. See nõuab omaette teadmisi konkreetsete domeenide ettenähtud tööpõhimõtetest ja tööd täpsete kontranäidete analüüsimiseks. Sellele vaatamata annavad läbiviidud testid suuna, kus tõenäoliselt probleemid võivad olla.



\end{document}