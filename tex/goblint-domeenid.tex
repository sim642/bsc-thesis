\documentclass[../thesis.tex]{subfiles}

\begin{document}

\section{Testitud Goblinti domeenid}
\label{app:goblint-domeenid}
\begin{textblock}{5.5}(0, 0)
Joonisel on toodud testitud Goblinti domeenid nende lähtekoodis~\cite{goblint_repo} olevate nimedega. Väljuvate nooltega on näidatud vastava domeeni sisemised komponendid teiste domeenide kujul.

Punktiirkastiga ümbritsetud domeenid moodustavad Goblintis mingi tervikliku nimetatud domeeni või analüüsi.

Hallis kirjas (ja sulgudes) on domeenid, millel puuduvad sisulised generaatorid.
Hallid nooled näitavad, kuidas on analüüsi domeen seotud eraldi testitud domeenidega.
\end{textblock}
\vspace{-7ex} % hack to fit both on one page
\begin{center}
\noindent
\begin{tikzpicture}[
	remember picture,
	level distance=1cm,
	sibling distance=3.5cm,
	edge from parent/.style={draw,-latex},
	on grid=true,
	narrow/.style={sibling distance=2cm},
	bot/.style={text=gray},
	group/.style={rectangle,draw=black,dashed},
	every label/.style={font=\itshape}
	%style={font=\ttfamily}
]
\node {}
	child { node (DeadCodeNorth) {Dom}
		child { node (DeadCodeSouth) {Lift}
			child { node {HConsed}
				child { node (PathSensitiveNorth) {Hoare}
					child { node {ToppedSet}
						child { node (PathSensitiveSouth) {SetDomain}
							child { node {DomList}
								% base
								child [xshift=-2cm] { node (BaseNorth) {Prod}
									child { node {CPA}
										child { node (BaseWest) {MapBot\_LiftTop}
											child { node {LiftTop}
												child { node {MapBot}
													child { node [bot] (PMap) {(PMap)} }
												}
											}
										}
									}
									child { node {Flag}
										child { node (BaseEast) {SimpleThreadDomain}
											child { node {ProdSimple}
												child { node {ProdConf}
													child [narrow] { node {Simple}
														child { node {Chain}
														}
													}
													child [narrow] { node {Flat}
														child { node (BaseSouthEast) {Thread}
															child [xshift=1.25cm] { node [bot] (Variables) {(Variables)} }
														}
													}
												}
											}
										}
									}
								}
								% escape
								child { node (EscapeNorth) {EscapedVars}
									child { node {ToppedSet}
										child { node (EscapeSouth) {SetDomain} }
									}
								}
								% mutex
								child { node (MutexNorth) {Lockset}
									child { node {Reverse}
										child { node {ToppedSet}
											child { node (MutexEast) {SetDomain}
												child { node {Prod}
													child { node (MutexProdConf) {ProdConf}
														child [xshift=-1cm] { node {NormalLat}
															child { node [bot] (MutexSouthWest) (Normal) {(Normal)} }
														}
													}
												}
											}
										}
									}
								}
							}
						}
					}
				}
			}
		}
	};

\path[-latex] (EscapeSouth) edge (Variables);

\node[group,fit=(DeadCodeNorth) (DeadCodeSouth),label=right:DeadCode] {};
\node[group,fit=(PathSensitiveNorth) (PathSensitiveSouth),label=right:PathSensitive] {};
\node[group,fit=(BaseNorth) (BaseWest) (BaseEast) (BaseSouthEast),label={[xshift=-2.5cm]below:Base}] {};
\node[group,fit=(EscapeNorth) (EscapeSouth),label={[xshift=0.6cm]below:Escape}] {};
\node[group,fit=(MutexNorth) (Normal) (MutexEast),label=below:Mutex] {};
\end{tikzpicture}
\vspace{1ex}
\textcolor{gray}{\hrule}
\vspace{2ex}
\begin{tikzpicture}[
	remember picture,
	level distance=1cm,
	sibling distance=2cm,
	edge from parent/.style={draw,-latex},
	on grid=true
	%style={font=\ttfamily}
]
\node (IntDomTuple) {IntDomTuple}
	child { node {Trier}
		child { node {SetDomain}
			child { node (Integers) {Integers} }
		}
		child { node (Interval32) {Interval32} }
	}
	child { node {} edge from parent [draw=none] }
	child { node {CircInterval} }
	child { node (Enums) {Enums} };
\path[-latex] (IntDomTuple) edge (Interval32)
	(Enums) edge (Integers)
			edge (Interval32);
\node[above=2cm of Integers,xshift=-2cm] (Flattened) {Flattened}
	child { node (Flat) {Flat} };
\node[left=2cm of Flattened] (Lifted) {Lifted}
	child { node (Lift) {Lift} };
\path[-latex] (Flat) edge (Integers)
	(Lift) edge (Integers);
\node[below=2cm of Enums] (Booleans) {Booleans};
\end{tikzpicture}
\begin{tikzpicture}[
	remember picture,
	overlay
]
\path[-latex] (MutexProdConf) edge[bend left=30] (Booleans);

\path[gray,-latex] (Normal) edge (IntDomTuple);
\path[gray,-latex,dashed] (PMap) edge [out=300,in=160] (IntDomTuple);
\end{tikzpicture}
\end{center}


\end{document}