\documentclass[../thesis.tex]{subfiles}

\begin{document}

\section{Domeeni omadused}
Selles peatükis tuuakse üldised abstraktsete domeenide omadused, mis kataks kõik universaalselt kontrollitavad aspektid. Nende hulka kuuluvad täielike võrede, programmianalüüsi spetsiifilised ja abstraktsiooni korrektsuse omadused.

\subsection{Võre omadused}
\label{sec:lattice-props}
Kuna nõutakse, et domeenid oleks täielikud võred, siis peavadki nad kõiksugu võrede tingimusi rahuldama. Järgnev on ülevaade erinevatest omadustest, mis täielikel võredel kehtima peaks:

\noindent
Olgu $\mathbb{D}$ täielik võre, siis iga $a, b, c \in \mathbb{D}$ korral peavad kehtima järgnevad tingimused:

\paragraph{Osalise järjestuse omadused} (definitsioonist~\ref{def:järjestatud_hulk})
\begin{itemize}[nosep]
	\item $a \sqleq a$ (refleksiivsus);
	\item kui $a \sqleq b$ ja $b \sqleq c$, siis $a \sqleq c$ (transitiivsus);
	\item kui $a \sqleq b$ ja $b \sqleq a$, siis $a = b$ (antisümmeetrilisus);
\end{itemize}

\paragraph{Rajade omadused}~\cite{might_orders}
\begin{itemize}[nosep]
	\item $a \sqleq a \sqcup b$ ja $b \sqleq a \sqcup b$ (definitsioonist~\ref{def:join});
	\item $a \sqcap b \sqleq a$ ja $a \sqcap b \sqleq b$ (definitsioonist~\ref{def:meet});
	\item kui $a \sqleq c$ ja $b \sqleq c$, siis $a \sqcup b \sqleq c$ (definitsioonist~\ref{def:join});
	\item kui $c \sqleq a$ ja $c \sqleq b$, siis $c \sqleq a \sqcap b$ (definitsioonist~\ref{def:meet});
\end{itemize}

\paragraph{Rajade tehete omadused}~\cites[6]{laan_voreteooria}[39]{davey_lattices}
\begin{itemize}[nosep]
	\item $(a \sqcup b) \sqcup c = a \sqcup (b \sqcup c)$ ja $(a \sqcap b) \sqcap c = a \sqcap (b \sqcap c)$ (assotsiatiivsus);
	\item $a \sqcup b = b \sqcup a$ ja $a \sqcap b = b \sqcap a$ (kommutatiivsus);
	\item $a \sqcup a = a$ ja $a \sqcap a = a$ (idempotentsus);
	\item $a \sqcup (a \sqcap b) = a$ ja $a \sqcap (a \sqcup b) = a$ (neelduvus);
\end{itemize}

\paragraph{Vähima ja suurima elemendi omadused}~\cite{might_orders}
\begin{itemize}[nosep]
	\item $\bot \sqleq a$;
	\item $a \sqleq \top$;
	\item $a \sqcup \bot = a$;
	\item $a \sqcap \top = a$;
\end{itemize}

\paragraph{Järjestuse ja rajade tehete seosed}
Järgnevad on ekvivalentsed~\cite[39]{davey_lattices}:
\begin{enumerate}[nosep]
	\item $a \sqleq b$,
	\item $a \sqcup b = b$,
	\item $a \sqcap b = a$.
\end{enumerate}


\subsection{Laiendamine ja kitsendamine}
\label{sec:widen-narrow}
Kui joonisel~\ref{fig:prog-if} olev programm on tsükliteta ja selle analüüsi saaks teostada ka ilma võrratuste süsteemi püsipunkti leidmata, siis üldiselt see nii pole. Huvitavamad programmid sisaldavad tsükleid, mille korral ülalkirjeldatud algoritm on analüüsimiseks vajalik. Olgu nüüd vaatluse all tsükliga programm jooniselt~\ref{fig:prog-while} ja domeen $\mathbb{D} = (\Var \to \mathbb{I})_\bot$ nagu viimati, siis võrratuste ja võrrandite süsteemid on järgnevad:
\[
	\begin{cases}
		x_1 \sqgeq \top \\
		\begin{comment}
		\begin{rcases*}
			x_2 \sqgeq \tf_{1,2}(x_1) \\
			x_2 \sqgeq \tf_{3,2}(x_3)
		\end{rcases*} \Rightarrow
		\end{comment}
		x_2 \sqgeq \tf_{1,2}(x_1) \sqcup \tf_{3,2}(x_3) \\
		x_3 \sqgeq \tf_{2,3}(x_2) \\
		x_4 \sqgeq \tf_{2,4}(x_2)
	\end{cases}
	\  \text{ja} \quad
	\begin{cases}
		x_1 = \top \\
		\begin{aligned}
			x_2 &= \tf_{1,2}(x_1) \sqcup \tf_{3,2}(x_3) = \\
			&= x_1[\texttt{x} \mapsto [0, 0]] \sqcup x_3[\texttt{x} \mapsto d(\texttt{x}) + [1, 1]]
		\end{aligned} \\
		x_3 = \tf_{2,3}(x_2) = x_2[\texttt{x} \mapsto d(\texttt{x}) \sqcap [-\infty, 99]] \\
		x_4 = \tf_{2,4}(x_2) = x_2[\texttt{x} \mapsto d(\texttt{x}) \sqcap [100, +\infty]]
	\end{cases}.
\]

Nende vähim püsipunkt (tabelis~\ref{tab:while-iter}) leitakse naiivse iteratsiooni 204. sammul. Itereerimisel sisuliselt tehakse läbi kõik 100 tsükli iteratsiooni, mistõttu sellise süsteemi püsipunkti leidmise ajaline keerukus on $O(k)$, kus $k$ on tsükli iteratsioonide arv. Selline ebaefektiivsus on vastuvõetamatu, sest toimub justkui programmi kogu töö simuleerimine, mis on vastuolus staatilise analüüsi eesmärgiga. Keerulisemate programmide puhul on isegi võimalik, et analüüs ei termineeru~\cite[60]{seidl_foundations}.

\begin{figure}
	\centering
	\begin{subfigure}[b]{0.4\textwidth}
		\centering
		\begin{bminted}{c}
			int x = 0;
			while (x < 100)
				x++;
		\end{bminted}
		\caption{C-keelne lähtekood}
	\end{subfigure}
	~
	\begin{subfigure}[b]{0.4\textwidth}
		\centering
		\begin{tikzpicture}[
			->,>=stealth,
			node distance=0.75cm,
			every state/.style={inner sep=3pt, minimum size=0pt},
			stmt/.style={font=\ttfamily},
			initial text={},
			initial where=above
		]
			\node[initial,state] (1) {1};
			\node[state] (2) [below=of 1] {2};
			\node[state] (3) [below left=of 2] {3};
			\node[accepting,state] (4) [below right=of 2] {4};

			\path
				(1) edge node[stmt,right] {x = 0} (2)
				(2) edge[bend right] node[stmt,above left,near start] {x < 100} (3)
				    edge node[stmt,above right] {x >= 100} (4)
				(3) edge[bend right] node[stmt,below right,very near start] {x++} (2);
		\end{tikzpicture}
		\caption{Juhtimisvoograaf}
	\end{subfigure}

	\caption{Tsükliga programmi näidis.}
	\label{fig:prog-while}
\end{figure}

\paragraph{Laiendamine}
Probleemi põhjuseks on asjaolu, et intervalldomeen ei rahulda kasvavate jadade stabiliseerumise tingimust (\ingl{ascending chains condition}), sest selles leiduvad lõpmatud rangelt kasvavad ahelad, näiteks
\[
	[0, 0] \sqlt [0, 1] \sqlt [0, 2] \sqlt \ldots \sqlt [0, +\infty].
\]
\TODO{Seletada kuidagi paremini? Mitte tuua sisse rangelt kasvavust $\sqlt$?}
Püsipunkti leidmise igal iteratsioonil tsüklis muutuja väärtuste lõik suureneb ühe võrra seetõttu stabiliseerumiseks läheb ebameeldivalt palju iteratsioone. Probleemi lahendamiseks muudetakse analüüsimist nii, et püsipunkti leidmine toimuks kiirendatult, tuues meelega sisse täiendava ebatäpsuse. Sellist võtet nimetatakse laiendamiseks (\ingl{widening}).

Lahendatav võrratus $\bar{x} \sqgeq \bar{f}(\bar{x})$ kirjutatakse samaväärsel akumuleerival kujul $\bar{x} = \bar{x} \sqcup \bar{f}(\bar{x})$, kus ülemraja asemele valitakse sobiv \emph{laiendamise operaator} $\widen$ ja otsitakse vähim püsipunkt võrrandile $\bar{x} = \bar{x} \widen \bar{f}(\bar{x})$. Intervallidel defineeritakse see järgnevalt:
\begin{gather*}
	[l_1, u_1] \widen [l_2, u_2] = [l, u], \text{ kus} \\
	l = \begin{cases}
		l_1, & \text{kui } l_1 \leq l_2, \\
		-\infty, & \text{muidu}
	\end{cases}
	\qquad \text{ja} \qquad
	u = \begin{cases}
		u_1, & \text{kui } u_2 \leq u_1, \\
		+\infty, & \text{muidu}.
	\end{cases}
\end{gather*}
Intervallide laiendamise idee seiseb selles, et intervall suurenedes suureneb kohe vastavas suunas lõpmatusse, garanteerides, et intervall saab suureneda ülimalt kaks korda, mis muudab rangelt kasvavad ahelad lõplikeks. Sellise modifikatsiooniga süsteemi vähim püsipunkt (tabelis~\ref{tab:while-widen}) leitakse juba naiivse iteratsiooni 6. sammul, mis ühest küljest on oluline võit, teisest kaotus, sest lahend pole nii täpne kui eelnevalt.

\paragraph{Kitsendamine}
Kuna $\bar{f}$ on monotoonne, siis selle rakendamine püsipunktile tulemust ebatäpsemaks ei muuda, aga võib ebatäpset laiendatud tulemust siiski parandada. Sellist võtet nimetatakse kitsendamiseks (\ingl{narrowing}) ning eelmisele laiendatud lahendile kitsendamist rakendades stabiliseerub tulemus (tabelis~\ref{tab:while-narrow}) 4. sammul, jõudes sama tulemuseni, mis ilma laiendamist ja kitsendamist. Üldiselt see siiski nii ei pruugi olla, kuid tulemuse täpsust suurendab ikka.

Kitsendamisel võib tekkida analoogiline probleem kahanevate jadade stabiliseerumise tingimusega (\ingl{descending chains condition}) ja lõpmatute rangelt kahanevate ahelatega, mistõttu funktsiooni korduv rakendamine mõistliku sammude arvuga ei stabiliseeru. Samas võib kitsendamist lõpetada mistahes hetkel ja analüüs on ikka korrektne. Kitsendamise stabiliseerumiseks kasutatakse analoogilist võtet: lahendatav võrratus kirjutatakse samaväärsel kujul $\bar{x} = \bar{x} \sqcap \bar{f}(\bar{x})$, kus alamraja asemele valitakse sobiv \emph{kitsendamise operaator} $\narrow$ ja otsitakse vähim püsipunkt võrrandile $\bar{x} = \bar{x} \narrow \bar{f}(\bar{x})$. Intervallidel defineeritakse see järgnevalt:
\begin{gather*}
	[l_1, u_1] \narrow [l_2, u_2] = [l, u], \text{ kus} \\
	l = \begin{cases}
		l_2, & \text{kui } l_1 = -\infty, \\
		l_1, & \text{muidu}
	\end{cases}
	\qquad \text{ja} \qquad
	u = \begin{cases}
		u_2, & \text{kui } u_1 = +\infty, \\
		u_1, & \text{muidu}.
	\end{cases}
\end{gather*}
Intervallide kitsendamise idee seiseb selles, et intervall vähenedeb ainult lõpmatusest, garanteerides, et intervall saab väheneda ülimalt kaks korda, mis muudab rangelt kahanevad ahelad lõplikeks. Uuritud näitel kiirendatud kitsendamine jõuab sama tulemuseni kui tavaline kitsendamine (tabelis~\ref{tab:while-narrow}). Üldiselt see nii ei pruugi olla, sest kiirendamisel tuuakse sisse meelega täiendav ebatäpsus.

\begin{table}
	\caption{Tsükliga näiteprogrammi (joonisel~\ref{fig:prog-while}) analüüsi lahendid erinevatel meetoditel.}
	\centering
	\begin{subtable}[t]{0.3\textwidth}
		\caption{Iteratsiooni lahend}
		\centering
		$\begin{tabu}{c|c}
		\hline
		 & \texttt{x} \\
		\hline
		x_1 & \top \\
		x_2 & [0, 100] \\
		x_3 & [0, 99] \\
		x_4 & [100, 100] \\
		\hline
		\end{tabu}$
		\label{tab:while-iter}
	\end{subtable}
	~
	\begin{subtable}[t]{0.3\textwidth}
		\caption{Laiendamise tulemus}
		\centering
		$\begin{tabu}{c|c}
		\hline
		 & \texttt{x} \\
		\hline
		x_1 & \top \\
		x_2 & [0, +\infty] \\
		x_3 & [0, +\infty] \\
		x_4 & [100, +\infty] \\
		\hline
		\end{tabu}$
		\label{tab:while-widen}
	\end{subtable}
	~
	\begin{subtable}[t]{0.3\textwidth}
		\caption{Kitsendamise tulemus}
		\centering
		$\begin{tabu}{c|c}
		\hline
		 & \texttt{x} \\
		\hline
		x_1 & \top \\
		x_2 & [0, 100] \\
		x_3 & [0, 99] \\
		x_4 & [100, 100] \\
		\hline
		\end{tabu}$
		\label{tab:while-narrow}
	\end{subtable}
	\label{tab:itersolve-while}
\end{table}

\paragraph{Omadused}
Laiendamise ja kitsendamise operaatorid peavad analüüsi korrektsuse tagamiseks rahuldama teatud tingimusi.

\noindent
Olgu $\mathbb{D}$ abstraktne domeen, siis iga $a, b \in \mathbb{D}$ korral peavad kehtima järgnevad tingimused:
\begin{itemize}[nosep]
	\item $a \sqcup b \sqleq a \widen b$ \cite[61]{seidl_foundations};
	\item $a \sqcap b \sqleq a \narrow b \sqleq a$ \cite[66]{seidl_foundations}.
\end{itemize}

Kusjuures need operaatorid pole tingimata assotsiatiivsed, kommutatiivsed, jne nagu nende algsed analoogid. Neid kasutatakse täpselt nii nagu ülal toodud, kus vasak operand on vana seisund ja parem operand on uus seisund. Sellel asjaolul põhineb nende defineerimine ja seda ka nähtud intervalldomeeni puhul.

\subsection{Abstraktsiooni korrektsus}
\label{sec:abstraction-props}
\TODO{\textit{Consistent abstract interpretations} -- Cousot. Korrektsus? Kooskõlalisus?}
Abstraktse domeeni defineerimisel tekib küsimus, kas saadud domeen abstraheerib võimalikke olukordi korrektselt, st käitub mingis mõttes sama moodi. Näiteks intervalldomeen peaks olema täisarvude alamhulkade domeeni abstraktsioon, mille tehted käituks nii, nagu nad käituks konkreetsete alamhulkadega. Sellist olukorda kirjeldab järgmine definitsioon:
\begin{definition}
\label{def:abstraction}
Domeeni $\mathbb{D}_2$ nimetatakse domeeni $\mathbb{D}_1$ \emph{korrektseks abstraktsiooniks}, kui leiduvad funktsioonid $\alpha: \mathbb{D}_1 \to \mathbb{D}_2$ ja $\gamma: \mathbb{D}_2 \to \mathbb{D}_1$, mis rahuldavad järgnevaid tingimusi~\cite[242]{cousot77}:
\begin{enumerate}[nosep]
	\item $\alpha$ on monotoonne, \label{item:alphamono}
	\item $\gamma$ on monotoonne,
	\item iga $x_2 \in \mathbb{D}_2$ korral $x_2 = \alpha (\gamma (x_2))$, \label{item:gamma}
	\item iga $x_1 \in \mathbb{D}_1$ korral $x_1 \sqleq \gamma (\alpha (x_1))$, \label{item:alpha}
	\item iga $x_1 \in \mathbb{D}_1$ korral $\alpha (f_1 (x_1)) \sqleq f_2 (\alpha (x_1))$, kus $f_1$ on monotoonne tehe domeenis $\mathbb{D}_1$ ja $f_2$ sellele vastav monotoonne tehe domeenis $\mathbb{D}_2$. \label{item:abstraction}
\end{enumerate}
\end{definition}
Funktsiooni $\alpha$ nimetatakse abstraktsioonifunktsiooniks (\ingl{abstraction function}) ja funktsiooni $\gamma$ konkretisatsioonifunktsiooniks (\ingl{concretization function}).
Seega võib rääkida, et $\mathbb{D}_1$ on konkreetne domeen ja $\mathbb{D}_2$ selle abstraktne domeen.
\TODO{Paremad tõlked?}

Tingimus~\ref{item:gamma} nõuab, et konkretiseerimine ei kaotaks täpsust.
Tingimus~\ref{item:alpha} lubab, et abstraheerimine kaotab täpsust (aga ei suurenda seda).
Tingimust~\ref{item:abstraction} illustreerib joonis~\ref{fig:abstraction} ja see tähendab, et abstraktse tehte teostamine abstraktsel väärtusel pole täpsem, kui konkreetse tehte teostamine konkreetsel väärtusel.

Viimast tingimust saab samaväärselt sõnastada $\gamma$ kaudu. Lisaks kogu definitsioonist järeldub domeenidevahline Galois' vastavus (\ingl{Galois connection}), millel on oluline roll abstraktse interpretatsiooni formaliseerimisel ja üleüldse võreteoorias~\cite{cousot77}.

\begin{figure}
	\centering
	\begin{tikzpicture}[
		on grid
	]
		\node (x1) {$x_1$};
		\node [right=4cm of x1] (f1x1) {$f_1(x_1)$};
		\node [below=2.5cm of x1] (ax1) {$\alpha(x_1)$};
		\node [right=4cm of ax1] (f2ax1) {$f_2(\alpha(x_1))$};
		\node [above=1cm of f2ax1] (af1x1) {$\alpha(f_1(x_1))$};

		\path[->]
			(x1) edge node [above] {$f_1$} (f1x1)
			     edge node [left,pos=0.2] {$\alpha$} (ax1)
			(f1x1) edge node [right,pos=0.4] {$\alpha$} (af1x1)
			(ax1) edge node [below] {$f_2$} (f2ax1);
		\path (af1x1) -- (f2ax1) node [midway,sloped] {$\sqleq$};

		\path[draw=gray,dashed] ($(x1)+(-3cm,-0.9cm)$) to node [above,pos=0.05] {$\mathbb{D}_1$} node [below,pos=0.05] {$\mathbb{D}_2$} ($(f1x1)+(2cm,-0.9cm)$); % TODO better positioning
	\end{tikzpicture}
	\caption{Definitsiooni~\ref{def:abstraction} tingimust~\ref{item:abstraction} illustreeriv skeem.}
	\label{fig:abstraction}
\end{figure}

Domeenide $\mathbb{D}_1 = \powerset{\mathbb{Z}}$ ja $\mathbb{D}_2 = \mathbb{I}_\bot$ korral~\cite[243]{cousot77}
\begin{align*}
	\alpha(x_1) &= \begin{cases}
		\bot, & \text{kui } x_1 = \emptyset, \\
		[\min x_1, \max x_1], & \text{muidu};
	\end{cases} \\
	\gamma(x_2) &= \begin{cases}
		\emptyset, & \text{kui } x_2 = \bot, \\
		\{l, \ldots, u\}, & \text{kui } x_2 = [l, u].
	\end{cases}
\end{align*}

\paragraph{Omadused}
Kui on teada, millist konkreetset domeeni mingi abstraktne domeen abstraheerib, ja nende vahel on defineeritud abstraheerimine ja konkretiseerimine, siis peavad kehtima definitsiooni~\ref{def:abstraction} tingimused.

Isegi kui konkretiseerimisfunktsioon leidub ja on matemaatiliselt kirjeldatud, siis ei pruugi olla võimalik seda analüsaatoris implementeerida, sest abstraktsele väärtusele vastav konkreetne väärtus võib olla ebapraktiliselt mahukas või isegi lõpmatu. Nii on näiteks ülaltoodud alamhulkade ja intervallide domeenide puhul, kus $\gamma([-\infty, +\infty]) = \mathbb{Z}$. Konkretiseerimisfunktsiooni puudumisel analüsaatori implementatsioonis pole loomulikult sellega seonduvat võimalik verifitseerida, kuid definitsiooni tingimusi~\ref{item:alphamono} ja~\ref{item:abstraction} saab kontrollida ikka. Seejuures ongi kasulik tingimuse~\ref{item:abstraction} toodud kuju, mitte selle konkretiseerimisega kuju.


\end{document}