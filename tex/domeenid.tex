\documentclass[../thesis.tex]{subfiles}

\begin{document}

\section{Abstraktsed domeenid}

\cite{laan}

\begin{definition}
\emph{Osaliselt järjestatud hulk} on paar $(A, \sqleq)$, kus $A$ on hulk, millel on defineeritud binaarne seos $\sqleq$, mis iga $a, b, c \in A$ korral rahuldab järgnevaid tingimusi:
\begin{itemize}[nosep]
	\item $a \sqleq a$ (refleksiivsus),
	\item kui $a \sqleq b$ ja $b \sqleq c$, siis $a \sqleq c$ (transitiivsus),
	\item kui $a \sqleq b$ ja $b \sqleq a$, siis $a = b$ (antisümmeetrilisus).
\end{itemize}
\end{definition}

\noindent Olgu $(A, \sqleq)$ järjestatud hulk ja $X \subseteq A$.

\begin{definition}
Elementi $c$ nimetatakse hulga $X$ \emph{ülemiseks tõkkeks}, kui iga $x \in X$ korral $x \sqleq c$. Vähimat ülemist tõket nimetatakse \emph{ülemiseks rajaks}, st $X$-i iga ülemise tõkke $d$ korral $c \sqleq d$.
\end{definition}

\begin{definition}
Elementi $c$ nimetatakse hulga $X$ \emph{alumiseks tõkkeks}, kui iga $x \in X$ korral $c \sqleq x$. Suurimat alumist tõket nimetatakse \emph{alumiseks rajaks}, st $X$-i iga alumise tõkke $d$ korral $d \sqleq c$.
\end{definition}

Hulga $X$ ülemist ja alumist raja tähistame vastavalt $\bigsqcup X$ ja $\bigsqcap X$. Kui $X = \{a, b\}$, siis tähistame ülemise ja alumise raja vastavalt $a \sqcup b$ ja $a \sqcap b$.

\begin{definition}
\emph{Täielik võre} on osaliselt järjestatud hulk, mille igal alamhulgal leidub ülemine ja alumine raja.
\end{definition}

Täielikus võres $(A, \sqleq)$ leidub vähim element $\bot = \bigsqcap A$ ja suurim element $\top = \bigsqcup A$.

\end{document}